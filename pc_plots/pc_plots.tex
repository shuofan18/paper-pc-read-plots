% !TeX program = pdfLaTeX
\documentclass[12pt]{article}
\usepackage{amsmath}
\usepackage{graphicx,psfrag,epsf}
\usepackage{enumerate}
\usepackage{natbib}
\usepackage{textcomp}
\usepackage[hyphens]{url} % not crucial - just used below for the URL
\usepackage{hyperref}
\providecommand{\tightlist}{%
  \setlength{\itemsep}{0pt}\setlength{\parskip}{0pt}}

%\pdfminorversion=4
% NOTE: To produce blinded version, replace "0" with "1" below.
\newcommand{\blind}{0}

% DON'T change margins - should be 1 inch all around.
\addtolength{\oddsidemargin}{-.5in}%
\addtolength{\evensidemargin}{-.5in}%
\addtolength{\textwidth}{1in}%
\addtolength{\textheight}{1.3in}%
\addtolength{\topmargin}{-.8in}%

%% load any required packages here


\usepackage{color}
\usepackage{fancyvrb}
\newcommand{\VerbBar}{|}
\newcommand{\VERB}{\Verb[commandchars=\\\{\}]}
\DefineVerbatimEnvironment{Highlighting}{Verbatim}{commandchars=\\\{\}}
% Add ',fontsize=\small' for more characters per line
\usepackage{framed}
\definecolor{shadecolor}{RGB}{248,248,248}
\newenvironment{Shaded}{\begin{snugshade}}{\end{snugshade}}
\newcommand{\KeywordTok}[1]{\textcolor[rgb]{0.13,0.29,0.53}{\textbf{#1}}}
\newcommand{\DataTypeTok}[1]{\textcolor[rgb]{0.13,0.29,0.53}{#1}}
\newcommand{\DecValTok}[1]{\textcolor[rgb]{0.00,0.00,0.81}{#1}}
\newcommand{\BaseNTok}[1]{\textcolor[rgb]{0.00,0.00,0.81}{#1}}
\newcommand{\FloatTok}[1]{\textcolor[rgb]{0.00,0.00,0.81}{#1}}
\newcommand{\ConstantTok}[1]{\textcolor[rgb]{0.00,0.00,0.00}{#1}}
\newcommand{\CharTok}[1]{\textcolor[rgb]{0.31,0.60,0.02}{#1}}
\newcommand{\SpecialCharTok}[1]{\textcolor[rgb]{0.00,0.00,0.00}{#1}}
\newcommand{\StringTok}[1]{\textcolor[rgb]{0.31,0.60,0.02}{#1}}
\newcommand{\VerbatimStringTok}[1]{\textcolor[rgb]{0.31,0.60,0.02}{#1}}
\newcommand{\SpecialStringTok}[1]{\textcolor[rgb]{0.31,0.60,0.02}{#1}}
\newcommand{\ImportTok}[1]{#1}
\newcommand{\CommentTok}[1]{\textcolor[rgb]{0.56,0.35,0.01}{\textit{#1}}}
\newcommand{\DocumentationTok}[1]{\textcolor[rgb]{0.56,0.35,0.01}{\textbf{\textit{#1}}}}
\newcommand{\AnnotationTok}[1]{\textcolor[rgb]{0.56,0.35,0.01}{\textbf{\textit{#1}}}}
\newcommand{\CommentVarTok}[1]{\textcolor[rgb]{0.56,0.35,0.01}{\textbf{\textit{#1}}}}
\newcommand{\OtherTok}[1]{\textcolor[rgb]{0.56,0.35,0.01}{#1}}
\newcommand{\FunctionTok}[1]{\textcolor[rgb]{0.00,0.00,0.00}{#1}}
\newcommand{\VariableTok}[1]{\textcolor[rgb]{0.00,0.00,0.00}{#1}}
\newcommand{\ControlFlowTok}[1]{\textcolor[rgb]{0.13,0.29,0.53}{\textbf{#1}}}
\newcommand{\OperatorTok}[1]{\textcolor[rgb]{0.81,0.36,0.00}{\textbf{#1}}}
\newcommand{\BuiltInTok}[1]{#1}
\newcommand{\ExtensionTok}[1]{#1}
\newcommand{\PreprocessorTok}[1]{\textcolor[rgb]{0.56,0.35,0.01}{\textit{#1}}}
\newcommand{\AttributeTok}[1]{\textcolor[rgb]{0.77,0.63,0.00}{#1}}
\newcommand{\RegionMarkerTok}[1]{#1}
\newcommand{\InformationTok}[1]{\textcolor[rgb]{0.56,0.35,0.01}{\textbf{\textit{#1}}}}
\newcommand{\WarningTok}[1]{\textcolor[rgb]{0.56,0.35,0.01}{\textbf{\textit{#1}}}}
\newcommand{\AlertTok}[1]{\textcolor[rgb]{0.94,0.16,0.16}{#1}}
\newcommand{\ErrorTok}[1]{\textcolor[rgb]{0.64,0.00,0.00}{\textbf{#1}}}
\newcommand{\NormalTok}[1]{#1}

\usepackage{longtable}
\usepackage{booktabs}

\begin{document}


\def\spacingset#1{\renewcommand{\baselinestretch}%
{#1}\small\normalsize} \spacingset{1}


%%%%%%%%%%%%%%%%%%%%%%%%%%%%%%%%%%%%%%%%%%%%%%%%%%%%%%%%%%%%%%%%%%%%%%%%%%%%%%

\if0\blind
{
  \title{\bf Can a deep learning model read residual plots?}

  \author{
        Shuofan Zhang \thanks{The authors gratefully acknowledge \ldots{}} \\
    Research Assistant, Monash University\\
     and \\     Dianne Cook \\
    Professor of Business Analytics, Monash University\\
      }
  \maketitle
} \fi

\if1\blind
{
  \bigskip
  \bigskip
  \bigskip
  \begin{center}
    {\LARGE\bf Can a deep learning model read residual plots?}
  \end{center}
  \medskip
} \fi

\bigskip
\begin{abstract}
Residuals plots are a primary means to diagnose statistical models. It
requires human evaluation to determine if the structure in the plot is
consistent with a random variation or not. If not, then the diagnosis is
that the model has not adequately captured the relationships between
response and explanatory variable in the data. This thesis develops a
computer vision model to read residual plots. It compares results with a
large database of human evaluations. The evaluations were conducted
using a protocol called the ``lineup'' which places residual plots in a
formal framework for statistical hypothesis testing. The comparison
between computer and human is made on a very restricted and controlled
set of residual plot structures. A new small human subject study is also
conducted to compare human vs.~computer in reading heteroscedasticity.
\end{abstract}

\noindent%
{\it Keywords:} Model Diagnosis, Computer Vision, Deep Learning
\vfill

\newpage
\spacingset{1.45} % DON'T change the spacing!

\section{Introduction}\label{introduction}

The derivation of hypothesis tests and their asymptotic distributions
constitute a considerable part of the statistics literature. However,
the derivation is often complex and the resulting test may lack power.
For example, in time series analysis, the commonly used unit root tests
all suffer from low power in distinguishing the unit root null from
stationary alternatives. In addition, data plots show a lot more
information than a single statistic. A good illustration would be
Anscombe's Quartet. \emph{``It is a set of four distinct data sets each
consisting of 11 (x,y) pairs where each dataset produces the same
summary statistics (mean, standard deviation and correlation) while
producing vastly different plots''} \citep{ANS73}. Matejka and
Fitzmaurice also did an interesting study on this issue, they used
`datasaurus' data from \citet{DS16} and generated a series of data with
same statistics but very different plots as shown in figure 1
\citep{JM17}. Instead of using single statistics, if the information in
the plots can be utilized in performing hypothesis tests, the power
(1-\(\beta\)) may be improved.

\begin{figure}
\centering
\includegraphics{pc_plots_files/figure-latex/saurus-1.pdf}
\caption{Each dataset has the same summary statistics to two decimal
places: (E(x)=54.26, E(y)=47.83, Pearson's r=, sd(x)=16.76, sd(y)=26.93}
\end{figure}

Former studies have shown that human eyes are sensitive to the
systematic patterns in data plots. With proper manipulation, visualized
plots can be used as test statistics and perform a valid hypothesis
test. One example of these protocols that provides inferential validity
is called ``lineup'' which was introduced by \citet{HW10}. \emph{``The
protocol consists of generating 19 null plots (could be other numbers),
inserting the plot of the real data in a random location among the null
plots and asking the human viewer to single out one of the 20 plots as
most different from the others''} \citep{HW10}. If the real plot is
chosen, it means the real data is likely to be different from the null
hypothesis, so we reject the null hypothesis with 5\% chance to be wrong
(Type I error). Because if all 20 plots are generated from the null
distribution, the chance of one plot being picked is \(1/20\). With the
assistance of ``lineup'', we avoid falling into the trap of apophenia
where we see patterns in random noise. This protocol has been proved to
be valid and powerful theoretically as well as practically through human
experiments \citep{MM13}. The human factors that may influence visual
statistical inference were also investigated by \citet{human2014}. The
experiments in \citet{human2014} suggest that \emph{``individual skills
vary substantially, but demographics do not have a huge effect on
performance.''} Although there are some statistically significant
factors such as ``having a graduate degree'' and ``living country'', the
effects of these factors are minimal. These results demonstrate the
robustness of the test against different human factors. Figure 2 is an
example of the lineup. Which plot do you think is the most different? If
you choose plot one, we are 95\% confident to reject the no-relationship
assumption between the two variables, ``hp'' and ``disp'' \citep{SIM18}.
The lineup protocol can also be used for other types of testing by
choosing different types of plot. For example, normality can be tested
using QQ plot; the difference in mean can be tested using box plot.

\begin{figure}
\centering
\includegraphics{pc_plots_files/figure-latex/lineup-1.pdf}
\caption{Scatterplot lineup example: one plot is the data, the rest are
generated from a null model assuming no relationship between the two
variables. In this lineup it is easy to see that plot 1, which is the
data plot, is different from the rest.}
\end{figure}

The question that arises today is whether the human evaluation can be
aided and supplemented by computer. If it is feasible, we can have a
computer process a lot more data than a human can manage. Thus, the cost
of rendering visual inference will become much lower. Therefore, in this
study, we introduce the computer vision technique to test the null of no
structure against linear patterns in a scatter plot, as an alternative
to the conventional \(t\)-test. To be more specific, the hypothesis we
are trying to test is:

\(H_0\): There are no relationships between the two variables in the
scatter plot.

\(H_1\): There is a linear relationship between the two variables in the
scatter plot where all Gauss-Markov assumptions are met.

The accuracy of the computer model is compared to both a database of
human evaluations from \citet{MM13} and the \(t\)-test. The motivation
for the task is provided in a blog post by Giora Simchoni \citep{SIM18}.
He has designed a deep learning model to test the significance of the
linear relationship between two variables for samples of size 50. The
model reached over 93\% accuracy on unseen test data. He also mentioned
that the computer failed to pick up a strong non-linear relationship
even though the Pearson'r is as high as -0.84 \citep{SIM18}. As Simchoni
explained in his article, the model can only distinguish linear
relationship from no-relationship as trained. However, we think this
fact is just another example reflecting the importance of visualization
as we discussed above. A strong correlation does not necessarily mean a
linear relationship. We should always refer to the plot before making
any statement. In addition, if we require the model to be more flexible,
it can be realised by adjusting the training design accordingly.

\section{Experimental Methods}\label{experimental-methods}

\subsection{The Human Evaluation}\label{the-human-evaluation}

A large database of results from human subjects was collected examining
the performance of the lineup protocol relative to classical tests. The
work is published in \citet{MM13}. This database forms the basis of the
test set used to examine the computer model performance.

In \citet{MM13}, \emph{``three experiments were conducted to evaluate
the effectiveness of the lineup protocol relative to the equivalent test
statistic used in the regression setting.''} In each experiment, they
simulated data from a controlled setting and then generated associated
lineup for the human to evaluate. The human subjects were hired from
Amazon Mechanical Turk, a marketplace for work that requires human
intelligence. \citep{amazon}

The controlled model in their first experiment is
\[Y_i=\beta_0+\beta_1 X_{i1}+\beta_2 X_{i2}+\epsilon_i\] where
\(\beta_0=5, \beta_1=15, X_1 \sim Poisson(\lambda=30), \epsilon_i\sim N(0,\sigma^2), i=1,2,...,n\),
\(\beta_2\) used in generating real data is specified in table 2.1.
While in the null model \(\beta_2=0\), and the null data was generated
by simulating from \(N(0,\hat{\sigma}^2)\). This experiment was aimed to
test the ability of human on detecting the effect of \(X_2\).

Their second experiment is very similar to the first one, but there is
only one continuous variable \(X_1\) on the right-hand side. It examined
the performance of humans in recognising linear association between two
variables, in direct comparison to conducting a \(t\)-test of
\(H_o: \beta_k=0\) vs \(H_a: \beta_k\neq 0\) assessing the importance of
including variable \(k\) in the linear model. The actual data model is
\[Y_i=\beta_0+\beta_1 X_{i1}+\epsilon_i\] where
\(\beta_0=6, X_1\sim N(0,1)\), and the null data was generated from
\(N(0, \hat{\sigma}^2)\).

The third experiment in their paper contains contaminated data where the
actual data were in fact generated from two different specifications.
\[Y_i=
  \begin{cases}
    \alpha+\beta X_i+\epsilon_i       & \quad X_i\sim N(0,1)\ \ i=1,...,n\\
    \lambda+\eta_i  & \quad X_i\sim N(\mu,1/3)\ \ i=1,...,n_c
  \end{cases}
\] where
\(\epsilon_i \sim N(0,\sigma), \eta_i \sim N(0,\sigma/3), \ \mu=-1.75, \ \beta\in(0.1, 0.4, 0.75, 1.25, 1.5, 2.25)\).
And \(n=100, n_c=15, alpha=0, \lambda=10, \sigma=3.5\). The null plots
were generated from \(N(0,\hat{\sigma}^2)\). Other parameters in the
``actual'' data sets of Turk experiment one and Turk experiment two are
shown in table 1.

\begin{longtable}[]{@{}cccc@{}}
\caption{Parameter values for simulation in Turk's
study.}\tabularnewline
\toprule
\begin{minipage}[b]{0.20\columnwidth}\centering\strut
Sample size (n)\strut
\end{minipage} & \begin{minipage}[b]{0.20\columnwidth}\centering\strut
Error SD(sigma)\strut
\end{minipage} & \begin{minipage}[b]{0.15\columnwidth}\centering\strut
Exp 1-beta2\strut
\end{minipage} & \begin{minipage}[b]{0.33\columnwidth}\centering\strut
Exp 2-beta1\strut
\end{minipage}\tabularnewline
\midrule
\endfirsthead
\toprule
\begin{minipage}[b]{0.20\columnwidth}\centering\strut
Sample size (n)\strut
\end{minipage} & \begin{minipage}[b]{0.20\columnwidth}\centering\strut
Error SD(sigma)\strut
\end{minipage} & \begin{minipage}[b]{0.15\columnwidth}\centering\strut
Exp 1-beta2\strut
\end{minipage} & \begin{minipage}[b]{0.33\columnwidth}\centering\strut
Exp 2-beta1\strut
\end{minipage}\tabularnewline
\midrule
\endhead
\begin{minipage}[t]{0.20\columnwidth}\centering\strut
100\strut
\end{minipage} & \begin{minipage}[t]{0.20\columnwidth}\centering\strut
5\strut
\end{minipage} & \begin{minipage}[t]{0.15\columnwidth}\centering\strut
0,1,3,5,8\strut
\end{minipage} & \begin{minipage}[t]{0.33\columnwidth}\centering\strut
0.25, 0.75, 1.25, 1.75, 2.75\strut
\end{minipage}\tabularnewline
\begin{minipage}[t]{0.20\columnwidth}\centering\strut
100\strut
\end{minipage} & \begin{minipage}[t]{0.20\columnwidth}\centering\strut
12\strut
\end{minipage} & \begin{minipage}[t]{0.15\columnwidth}\centering\strut
1,3,8,10,16\strut
\end{minipage} & \begin{minipage}[t]{0.33\columnwidth}\centering\strut
0.5, 1.5, 3.5, 4.5, 6\strut
\end{minipage}\tabularnewline
\begin{minipage}[t]{0.20\columnwidth}\centering\strut
300\strut
\end{minipage} & \begin{minipage}[t]{0.20\columnwidth}\centering\strut
5\strut
\end{minipage} & \begin{minipage}[t]{0.15\columnwidth}\centering\strut
0,1,2,3,5\strut
\end{minipage} & \begin{minipage}[t]{0.33\columnwidth}\centering\strut
0.1, 0.4, 0.7, 1, 1.5\strut
\end{minipage}\tabularnewline
\begin{minipage}[t]{0.20\columnwidth}\centering\strut
300\strut
\end{minipage} & \begin{minipage}[t]{0.20\columnwidth}\centering\strut
12\strut
\end{minipage} & \begin{minipage}[t]{0.15\columnwidth}\centering\strut
1,3,5,7,10\strut
\end{minipage} & \begin{minipage}[t]{0.33\columnwidth}\centering\strut
0, 0.8, 1.75, 2.3, 3.5\strut
\end{minipage}\tabularnewline
\bottomrule
\end{longtable}

Their second experiment utilized 70 lineups of size 20 plot, with
varying degrees of departure from the \(H_o: \beta_k=0\). There were 351
evaluations by human subjects. These results will be used in this study
for comparison with the deep learning model. An example lineup question
in Turk experiment 2 is shown in Figure \ref{expt2}. For this lineup, 63
of the 65 people who examined it selected the data plot (position 20)
from the null plots. There is clear evidence that the data displayed in
plot 20 is not from \(H_o: \beta_k=0\). The main procedures of the human
evaluating lineup are given in figure \ref{dghm}. ``Real data'' and
``null data'' stand for datasets simulated under the alternative
hypothesis and the null hypothesis respectively.

\begin{figure*}[h]
\centerline{\includegraphics[width=15cm]{figures/plot_turk2_300_350_12_3.png}}
\caption{One of 70 lineups used in experiment 2 Majumder et al (2012). Of the 65 people who examined the lineup,  63 selected the data plot, which is in position 20.}
\label{expt2}
\end{figure*}

\begin{figure*}[h]
\centerline{\includegraphics[width=15cm]{figures/diaghm.png}}
\caption{Diagram illustrating the process of human subject evaluation of lineups, and how performance is computed.}
\label{dghm}
\end{figure*}

\subsection{The Computer Model}\label{the-computer-model}

The model we use is the convolutional neural networks, also known as
convnets, a type of deep-learning model \emph{``almost universally used
in computer vision applications''} \citep{DLR18}. The very first
convolutional neural networks, called the ``LeNet5'' which was born in
1994, propelled the field of deep learning. However, this technique was
in incubation from 1998 to 2010. In recent years, with the increasing
data availability and more advanced technology, the design of the neural
network architecture became more and more successful. Many types of
neural network architectures have been developed since then, such as the
``Dan Ciresan Net'' which enabled the implementation of GPU for the
first time and the ``AlexNet'', which used the so-called ``ReLU''
function as the activation function and started a small revolution in
the deep learning world. \citep{cnn2017} Basically convolutional neural
networks has two interesting properties: ``the patterns they learn are
translation invariant'', and ``they can learn spatial hierarchies of
patterns'' \citep{DLR18}. The first property implies that once the model
learns how to recognize the linear patterns, it can detect these
patterns regardless of their direction, thus handling negative/positive
relationship automatically in our case. The architecture used in this
study is a fundamental one.

Unlike the classical programming where human input rules, in deep
learning paradigm, we provide data and the answers associated with the
data. Deep learning algorithm will output the rules, and these rules can
then be used on new data to make predictions. One can think of the deep
learning neural network as a complex nonlinear model which could
estimate millions of parameters (\(\textbf{w}\)) with a big enough
dataset. As usual regression problem, to get the estimates of unknown
parameters (\(\textbf{w}\)), we need to provide the model with the
dependent variable (\(y_i\)) and the independent variables
(\(\textbf{x}_i\)). In this case, the independent variable will be the
images of data plots (in forms of matrices) simulated from the null
distribution and the alternative distribution, and dependent variable
will be the labels of that plot indicating the true relationship of the
original data. The estimation method for the deep-learning model is
called ``backpropagation'' algorithm which is a way to train chains of
parametric operations using gradient-descent optimization. The
gradient-descent optimizer is meant to find the set of parameters such
that the cost function reaches its minimum. The form of the cost
functions or loss function is determined per each question. In this
paper, the deep learning model is expected to complete binary
classification task, detect ``linearly correlated variables'' from
``unrelated variables''. As introduced by \citet{DLR18},
\emph{``crossentropy is usually the best choice (as the loss function)
when you're dealing with models that output probabilities''}. Originated
from Information Theory, crossentropy is a quantity measuring the
distance between probability distributions. In deep learning world, it
measures the distance between the true distribution and the predictions.
Therefore, in this paper, the binary crossentropy loss function will be
used. The associated cost function is of the form,
\[J(\textbf{w})=- \frac{1}{N}\sum_{i=1}^N\left(  
\ y_i\ log\hat{y_i}+(1-y_i)\ log(1-\hat{y_i})  
\right)\] where
\(\hat{y_i} = g(\textbf{w} \times \textbf{x}_i) = \frac{1}{1+e^{-\textbf{w} \times \textbf{x}_i}}\)
and \(g(z)\) is the logistic function. Once we have the estimated
parameters (\(\hat{\textbf{w}}\)), we then can use them to classify
unseen data plots, e.g.~to perform a hypothesis test.

The main procedures involved in constructing and selecting a convnets
model are shown in figure \ref{dgpc}. The convnets is trained on
``train'' and ``validation'' set. A certain number of iterations over
all samples are done, the fitted convnets given by each iteration are
saved, one best model is chosen as the representative for the computer
according to the overall accuracy on the unseen ``test'' set.

All convolutional neural network related work is done by the Keras
\citep{keras} package in R \citep{R}, which interfaces to the python
software. The R package ggplot2 \citep{ggplot2} is used to generate the
plots. All plots are resized to \(150\times 150\) pixel and saved as
png. This size is similar to the plot size used in the lineup for human
evaluation.

\begin{figure*}[h]
\centerline{\includegraphics[width=15cm]{figures/diagpc.png}}
\caption{Diagram illustrating the training, diagnosis and choice of the computer model. Based on 480,000 simulated data sets used to create $150\times 150$ pixel images, divided into train, validation and test sets.}
\label{dgpc}
\end{figure*}

\emph{``A convnets takes as input tensors of shape (image height, image
width, image channels)''} \citep{DLR18}. The channels are normally equal
to three for RGB. In our case, the input tensors are of shape
\(150 \times 150 \times 1\). The channel is equal to one because the
input data is grayscale images. Therefore, the convnet will be
configured to process inputs of size (150, 150, 1). We'll do this by
passing the argument input\_shape = c(150, 150, 1) to the first layer.
The R codes below are used to build the convnets in R. From figure
\ref{modelstruc}, the output shape changes after every layer of ``conv''
and ``pooling'' operations. The original \(150 \times 150 \times 1\)
image is finally sliced into a \(7 \times 7 \times 128\) object (3D
tensor). The figure \ref{diagconv} describes how ``convolution'' and
``max pooling'' operation works. By ``convolution'' the image matrix is
multiplied by a filter, different filter gives different output matrix
which extracting different features from the image. Max pooling select
the max number within a certain area. Then we need to flatten these 3D
tensor into 1D tensor so that they can be processed by the ``sigmoid''
function in the end. The ``sigmoid'' is, in fact, a special case of
logistic function. \(S(\textbf{x})=\frac{1}{1+e^{-\textbf{x}}}\). From
this model structure, we can also see that a total number of 3,452,545
parameters need to be estimated, this is done by gradient descent. 10
epochs (1 epoch = 1 iteration over all samples) are done for training in
our first experiment. The model specifications given by each epoch are
saved, the one gives the overall highest accuracy is chosen to represent
the computer.

\begin{Shaded}
\begin{Highlighting}[]
\KeywordTok{library}\NormalTok{(keras)}
\NormalTok{model <-}\StringTok{ }\KeywordTok{keras_model_sequential}\NormalTok{() }\OperatorTok
\StringTok{  }\KeywordTok{layer_conv_2d}\NormalTok{(}\DataTypeTok{filters =} \DecValTok{32}\NormalTok{, }\DataTypeTok{kernel_size =} \KeywordTok{c}\NormalTok{(}\DecValTok{3}\NormalTok{, }\DecValTok{3}\NormalTok{), }
                \DataTypeTok{activation =} \StringTok{"relu"}\NormalTok{,}
                \DataTypeTok{input_shape =} \KeywordTok{c}\NormalTok{(}\DecValTok{150}\NormalTok{, }\DecValTok{150}\NormalTok{, }\DecValTok{1}\NormalTok{)) }\OperatorTok
\StringTok{  }\KeywordTok{layer_max_pooling_2d}\NormalTok{(}\DataTypeTok{pool_size =} \KeywordTok{c}\NormalTok{(}\DecValTok{2}\NormalTok{, }\DecValTok{2}\NormalTok{)) }\OperatorTok
\StringTok{  }\KeywordTok{layer_conv_2d}\NormalTok{(}\DataTypeTok{filters =} \DecValTok{64}\NormalTok{, }\DataTypeTok{kernel_size =} \KeywordTok{c}\NormalTok{(}\DecValTok{3}\NormalTok{, }\DecValTok{3}\NormalTok{), }
                \DataTypeTok{activation =} \StringTok{"relu"}\NormalTok{) }\OperatorTok
\StringTok{  }\KeywordTok{layer_max_pooling_2d}\NormalTok{(}\DataTypeTok{pool_size =} \KeywordTok{c}\NormalTok{(}\DecValTok{2}\NormalTok{, }\DecValTok{2}\NormalTok{)) }\OperatorTok
\StringTok{  }\KeywordTok{layer_conv_2d}\NormalTok{(}\DataTypeTok{filters =} \DecValTok{128}\NormalTok{, }\DataTypeTok{kernel_size =} \KeywordTok{c}\NormalTok{(}\DecValTok{3}\NormalTok{, }\DecValTok{3}\NormalTok{), }
                \DataTypeTok{activation =} \StringTok{"relu"}\NormalTok{) }\OperatorTok
\StringTok{  }\KeywordTok{layer_max_pooling_2d}\NormalTok{(}\DataTypeTok{pool_size =} \KeywordTok{c}\NormalTok{(}\DecValTok{2}\NormalTok{, }\DecValTok{2}\NormalTok{)) }\OperatorTok
\StringTok{  }\KeywordTok{layer_conv_2d}\NormalTok{(}\DataTypeTok{filters =} \DecValTok{128}\NormalTok{, }\DataTypeTok{kernel_size =} \KeywordTok{c}\NormalTok{(}\DecValTok{3}\NormalTok{, }\DecValTok{3}\NormalTok{), }
                \DataTypeTok{activation =} \StringTok{"relu"}\NormalTok{) }\OperatorTok
\StringTok{  }\KeywordTok{layer_max_pooling_2d}\NormalTok{(}\DataTypeTok{pool_size =} \KeywordTok{c}\NormalTok{(}\DecValTok{2}\NormalTok{, }\DecValTok{2}\NormalTok{)) }\OperatorTok
\StringTok{  }\KeywordTok{layer_flatten}\NormalTok{() }\OperatorTok
\StringTok{  }\KeywordTok{layer_dense}\NormalTok{(}\DataTypeTok{units =} \DecValTok{512}\NormalTok{, }\DataTypeTok{activation =} \StringTok{"relu"}\NormalTok{) }\OperatorTok
\StringTok{  }\KeywordTok{layer_dense}\NormalTok{(}\DataTypeTok{units =} \DecValTok{1}\NormalTok{, }\DataTypeTok{activation =} \StringTok{"sigmoid"}\NormalTok{)}
\end{Highlighting}
\end{Shaded}

\begin{figure*}[h]
\centerline{\includegraphics[width=15cm]{figures/modelstruc.png}}
\caption{The deep learning model structure used for the experiments.}
\label{modelstruc}
\end{figure*}

\begin{figure*}[h]
\centerline{\includegraphics[width=15cm]{figures/diagconv.png}}
\caption{Illustration of convolution and pooling steps on an image. The convolution step applies a fixed number of filters to sliding windows of $3\times 3$ cells. Pooling applies a statistic to distinct $2\times 2$ tiling of the image. In our model, the statistic used is the maximum of the four values. These transformations are the pre-processing steps done on every image in the training sample, to fit the model, and also to the validation and test images prior to prediction. }
\label{diagconv}
\end{figure*}

As for the time needed for computer training, since a large number of
images are employed in this study, and we rely only on the CPU, 10-20
hours is required for generating and saving all images, another 10-20
hours will be necessary for the convnets model to be trained and tested.
With an NVIDIA GPU, this duration will be shortened significantly.

\subsection{Simulation}\label{simulation}

In total 240,000 data sets are simulated under each the null and the
alternative hypothesis. 100,000 of them are set apart for training.
Another 40,000 of the data sets are set apart as the validation set in
order to monitor during training the accuracy of the model on data it
has never seen before. And the rest 100,000 data sets become the unseen
test set. We make the test set relatively large that we can compare the
performance of the convnet with the \(t\)-test with small variations. As
for the labels given to each image, we use the true population as the
samples' identification directly.

\subsubsection{Under the Alternative
Hypothesis}\label{under-the-alternative-hypothesis}

For ease of exposition, the plots used for training and testing is the
scatterplot between the dependent variable \(Y\) and the independent
variable \(X\). It can also be considered as the residual plot of such
data fitting to a constant model. Although only the simple regression
model is considered in this paper, but many of the results can be
generalized to other cases including multiple regression models. Because
the ``statistics'' we use is the scatterplot, in terms of teaching the
computer to recognize linear patterns from others, one variable is
enough to generate different patterns. And this makes the design process
much simpler.

The design for our model under the alternative hypothesis is similar to
what \citet{SIM18} did in his blog. But the parameters are tailored to
compare the computer performance with the Turk study results.

The model under the alternative is designed as:
\[Y_i = \beta_0 + \beta_1 X_{i}  + \varepsilon_i, ~~i=1, \dots , n\] And
all the parameters in our model were designed to cover the range used in
the second experiment in Turk study \citep{MM13}. Therefore, the
relevant parameters in our model are generated using the following
specification.

\begin{itemize}
\item
  \(X \sim N[0,\ 1]\)\\
  Distributions of X has an impact on the shape of the scatters. For
  instance, if X is generated from a uniform distribution, then the
  plots will look like a square when the sample size is large; while
  looking like a circle if X follows a normal distribution.
\item
  \(\beta_0 = 0\)\\
  Intercept is arbitrarily set to be zero because it has no impact on
  the patterns in the data plots.
\item
  \(\beta_1\sim U[-10, -0.1] \bigcup [0.1, 10]\)\\
  \(\beta_1\) is designed to be uniformly generated from -10 to 10
  (excluding -0.1 to 0.1).
\item
  \(\varepsilon\sim N(0, \sigma^2) \ where\ \sigma \sim U[1,12]\)\\
  \(\varepsilon\) is designed to be uniformly distributed from 1 to 12.
\item
  \(n=U[50,500]\)\\
  The sample sizes of each data set vary from 50 to 500 observations.
\end{itemize}

Figure \ref{fig:linear} shows four example plots generated using the
specifications above. To facilitate the computer vision, all texts,
ticks, and titles of X and Y axes are removed, so does the background
grid. Under this controlled structure, a total number of 240,000
datasets are simulated. Figure \ref{fig:simplot} contains a histogram of
the simulated n, a histogram of the simulated \(\beta\), a histogram of
the simulated \(\sigma\), a histogram of the estimated sample p-value, a
scatter plot of \(\beta\) against n and a scatter plot of \(\sigma\)
against n. These plots show good coverage over the alternative parameter
space.

\begin{figure}
\centering
\includegraphics{pc_plots_files/figure-latex/linear-1.pdf}
\caption{Four examples of data plots generated from the classic linear
model.}
\end{figure}

\begin{figure}
\centering
\includegraphics{pc_plots_files/figure-latex/simplot-1.pdf}
\caption{Overview of parameter values used in the linear class
simulation, for computer model training. Good coverage is obtained
across the parameter space.}
\end{figure}

\subsubsection{Under the Null
Hypothesis}\label{under-the-null-hypothesis}

When the two variables under tested are independent of each other, then
the data plots will not show any systematic patterns theoretically. It
is true that there must be some undesired patterns formed out of
randomness, especially when the sample size is small. Unlike what
Simchoni did in his post, no conventional tests will be used to sort out
the ``significant/insignificant'' samples. Because the answer to the
question that if the deep learning model can distinguish from patterns
formed by chance and by nature is also interesting. The model is
designed the same as the linear model:
\[Y_i = \beta_0 + \beta_1 X_{i}  + \varepsilon_i, ~~i=1, \dots , n\]
with elements of the model generated using the same specification as the
linear model, except\\
- \(\beta_1 = 0\)

Hence, the coefficient of \(X_i\) is always zero. So \(X\) and \(Y\) are
uncorrelated of each other.

Figure \ref{fig:norela} are four example plots generated using the
specifications above. Same as the linear model simulation, a total
number of 240,000 datasets are simulated under this structure. Figure
\ref{fig:simp} contains a histogram of the simulated n, a histogram of
the simulated \(\sigma\), a histogram of the estimated sample p-value
and a scatter plot of \(\sigma\) against n. These plots show good
coverage over the null parameter space. All simulated data and
associated parameters including estimated sample p-values of t-test are
saved and are used later on for calculating the performance of
conventional t-test.

\begin{figure}
\centering
\includegraphics{pc_plots_files/figure-latex/norela-1.pdf}
\caption{Four examples of data plots generated with two independent
variables}
\end{figure}

\begin{figure}
\centering
\includegraphics{pc_plots_files/figure-latex/simp-1.pdf}
\caption{Overview of parameter values used in the null class simulation,
for computer model training. Good coverage is obtained across the
parameter space.}
\end{figure}

\section{Results and Discussion}\label{results-and-discussion}

The plot of the training history in figure \ref{histlinear} shows high
accuracy achieved in both train and validation set (93\%-94\%); slight
overfitting starts from the fourth epoch; the variation of the values of
accuracy and loss in validation set are very small after the fourth
epoch. Hence, both our convnets and dataset are large enough and the
training of our first experiment can be concluded. Then we select the
fourth, sixth, eighth and the tenth model to have them tested on the
unseen test set. And the results are shown in the table
\ref{checkpoints}. In this table, the ``\(1-\alpha\)'' means the
accuracy of each computer model tested on the ``null data'' in the test
set only. \(\alpha\) here is an analogy to the Type I error in the
conventional hypothesis test. Similarly, the ``power'' is the accuracy
of each computer model tested on the ``linear data'' in the test set
only. The t-test performance in this table is calculated at 5\%
significance level.

\begin{figure*}[h]
\centerline{\includegraphics[width=15cm]{figures/linear_history_plot.png}}
\caption{Training and validation metrics of linear vs. null model in our first experiment. The top plot is the accuracy achieved in train and validaton sets, while the bottom plot is the loss. Red presents model performance in train set while green presents validation.}
\label{histlinear}
\end{figure*}

\begin{table}[ht]
\begin{center}
\begin{tabular}{|l|rrr|}\hline
Tests & Linear & Null & Overall \\\hline
4 epoch & 0.892 & 0.984 &  0.938 \\   
6 epoch & 0.889 & 0.986 & 0.937 \\
8 epoch & 0.896 & 0.981 & 0.939 \\
10 epoch & 0.904 & 0.971 & 0.938 \\
5\% $t$-test & 0.921 & 0.949 & 0.935\\\hline
\end{tabular}
\end{center}
\caption{Performance of four checkpoints from the {\em convnets} model, and the 5\% significant $t$-test, computed on the test set. Accuracy is reported for each class, and overall. There is a slight improvement as the number of epochs increases, with 10 epochs being reasonably close to the ideal $t$-test accuracy.}
\label{checkpoints}
\end{table}

The 8th model is chosen according to the overall accuracy on the test
set. We should note that since the majority of the real data plots in
Turk's experiment have been generated from linear relationships (when
the alternative hypothesis is true), it is a disadvantage for the
computer comparing in terms of being tested on the Turk's data. Because
the \(\alpha\) is approximately 2\% for the 8th computer model, the 5\%
significant t-test and 5\% human evaluations may have higher power than
the computer model.

The performance of the computer model for the Turk study data is tested
in three steps:

\begin{itemize}
\item
  Re-generate the 70 ``real plots'' using the same data in Turk study
  (without null plots);
\item
  Create a separate test directory for the 70 ``real plots''
  exclusively;
\item
  The computer model's predicted accuracy over the 70 ``real plots'' is
  recorded as the model's performance.
\end{itemize}

The conclusion of human evaluation is obtained differently from the
computers. Because human evaluated ``lineup'', not only the ``real
plots''. The performance is tested in five steps:

\begin{itemize}
\item
  Count the total number of evaluations made by human for one lineup (N)
  and the number of correct answers for that lineup (k);
\item
  Obtain N and k for all 70 lineups;
\item
  Calculate p-value associated with each real plot using the formula
  introduced in section 2 of \citet{MM13};
\item
  Draw the conclusion: reject the null when the calculated p-value is
  smaller than \(\alpha\).
\item
  The accuracy of the conclusions the 70 real plots is presenting for
  the human performance.
\end{itemize}

For a fair competition, the Type I error (\(\alpha\)) should be held the
same for all test methods. However, we do not have direct control over
the \(\alpha\) of the computer model. Because the \(\alpha\) estimated
from the computer model is close to 2\%. Therefore, 2\% significant
t-test and 2\% significant human conclusion are also included to give a
complete picture of the comparison. The comparing result given in table
3 is interesting. Human achieves the highest accuracy, and the
conclusion from the human evaluation is robust to smaller p-values; 5\%
significant t-test is the second best, 2\% significant t-test and the
computer model give the same results.

\begin{longtable}[]{@{}cccc@{}}
\caption{Accuracy of testing the 70 data plots evaluated by human
computer and the conventional t-test.}\tabularnewline
\toprule
\begin{minipage}[b]{0.09\columnwidth}\centering\strut
Rank\strut
\end{minipage} & \begin{minipage}[b]{0.17\columnwidth}\centering\strut
Tests\strut
\end{minipage} & \begin{minipage}[b]{0.21\columnwidth}\centering\strut
No. of correct\strut
\end{minipage} & \begin{minipage}[b]{0.12\columnwidth}\centering\strut
Accuracy\strut
\end{minipage}\tabularnewline
\midrule
\endfirsthead
\toprule
\begin{minipage}[b]{0.09\columnwidth}\centering\strut
Rank\strut
\end{minipage} & \begin{minipage}[b]{0.17\columnwidth}\centering\strut
Tests\strut
\end{minipage} & \begin{minipage}[b]{0.21\columnwidth}\centering\strut
No. of correct\strut
\end{minipage} & \begin{minipage}[b]{0.12\columnwidth}\centering\strut
Accuracy\strut
\end{minipage}\tabularnewline
\midrule
\endhead
\begin{minipage}[t]{0.09\columnwidth}\centering\strut
1\strut
\end{minipage} & \begin{minipage}[t]{0.17\columnwidth}\centering\strut
Human 5\%\strut
\end{minipage} & \begin{minipage}[t]{0.21\columnwidth}\centering\strut
47\strut
\end{minipage} & \begin{minipage}[t]{0.12\columnwidth}\centering\strut
0.6714\strut
\end{minipage}\tabularnewline
\begin{minipage}[t]{0.09\columnwidth}\centering\strut
1\strut
\end{minipage} & \begin{minipage}[t]{0.17\columnwidth}\centering\strut
Human 2\%\strut
\end{minipage} & \begin{minipage}[t]{0.21\columnwidth}\centering\strut
47\strut
\end{minipage} & \begin{minipage}[t]{0.12\columnwidth}\centering\strut
0.6714\strut
\end{minipage}\tabularnewline
\begin{minipage}[t]{0.09\columnwidth}\centering\strut
2\strut
\end{minipage} & \begin{minipage}[t]{0.17\columnwidth}\centering\strut
T-test 5\%\strut
\end{minipage} & \begin{minipage}[t]{0.21\columnwidth}\centering\strut
43\strut
\end{minipage} & \begin{minipage}[t]{0.12\columnwidth}\centering\strut
0.6143\strut
\end{minipage}\tabularnewline
\begin{minipage}[t]{0.09\columnwidth}\centering\strut
3\strut
\end{minipage} & \begin{minipage}[t]{0.17\columnwidth}\centering\strut
Computer 2\%\strut
\end{minipage} & \begin{minipage}[t]{0.21\columnwidth}\centering\strut
39\strut
\end{minipage} & \begin{minipage}[t]{0.12\columnwidth}\centering\strut
0.5571\strut
\end{minipage}\tabularnewline
\begin{minipage}[t]{0.09\columnwidth}\centering\strut
4\strut
\end{minipage} & \begin{minipage}[t]{0.17\columnwidth}\centering\strut
T-test 2\%\strut
\end{minipage} & \begin{minipage}[t]{0.21\columnwidth}\centering\strut
39\strut
\end{minipage} & \begin{minipage}[t]{0.12\columnwidth}\centering\strut
0.5571\strut
\end{minipage}\tabularnewline
\bottomrule
\end{longtable}

Under the condition specified for simulation, the conventional t-test is
known to be the uniformly most powerful (UMP) test in terms of detecting
the linear relationship according to the Neyman--Pearson lemma.
\citep{Neyman289} Although human achieved the best performance in the
Turk experiment dataset, it does not mean computer does badly since the
Turk experiment dataset only contains 70 plots. As we can see from table
\ref{checkpoints}, t-test and convnets behave quite similarly on both
the test set and the Turk's experiment data.

To check if the \(t\)-test and the convnets do perform similarly, we
calculated the accuracy of the \(t\)-test again, with different
\(\alpha\), from 0.005 to 0.1 with 0.005 increments, on all 200,000 test
sets. The estimated power and overall accuracy were recorded. When
\(\alpha=0.015\), the overall accuracy reaches its maximum. This value
approximately coincides with the \(\alpha\) chosen by the convnets. And
since the \(\alpha\) of convnets is from 0.0142 to 0.0347 on the test
set, we truncated the t-test data to create figure \ref{fig:ttdl}. The
upper dots represent overall accuracy achieved by convnets (red) and
t-test (green), while the lower dots stand for the estimated power in
the test set. The smooth line overlaid aids the eye in seeing patterns.
From this graph, we can see the convnets and t-test perform very
similarly, while t-test has overall better performance.

Given the size of our test set (200,000 images totally), it is
reasonable to assume that the convnets is, in fact, trying to approach
the t-test. In other words, the best strategy the convnets learned, in
this case, may be close to the \(t\)-test, the UMP.

\begin{figure}

{\centering \includegraphics{pc_plots_files/figure-latex/ttdl-1} 

}

\caption{Comparison between computer model and t-test for alpha in (0.01, 0.04), they perform very similarly, but t-test has overall better performance.}\label{fig:ttdl}
\end{figure}

In summary, performance of the convnets on testing the linear vs no
structure is comparable to both the human subjects' results and the
\(t\)-test.

Although constructing the assumptions used for simulation and the
implications for violation maybe challenging. This study gives hope to
the future utilization of convnets in reading the residual plot. Unlike
the conventional distribution tests, convnets does not require
convoluted mathematical derivations and can be applied to any
visualizable problems. Compared to human, the convnets is more stable in
that it always gives consistent prediction once it is well trained while
different groups of people may have different opinions on one plot. In
addition, the computer training and testing can be done by a single
computer which barely costs anything (other than our genuine efforts)
while human evaluations could be much more expensive.

On the other hand, the computer also has pitfalls. Its ability is
limited to the patterns provided by the training. The more patterns we
want the convnets to recognize, the more structures we need to feed it.
As for the future study, more types of structure could be considered,
for example, testing heteroskedasticity, non-linear relationship,
outliers, etc. The binary classification can also be extended into a
multiclass classification, e.g.~one convnets can be trained to recognize
several departures from the null. What's more, to provide a more
reliable comparison between human and computers, a larger human
experiment is required. For an even further step, the convnets can be
designed for even more general hypothesis testing purposes, the biggest
challenge would be finding the most appropriate plot (the test
statistic) which can show the key features for certain tests. For
instance, if we are interested in telling a time series data with unit
root from a trend stationary one, the most suitable plot may be the time
plot.

\bibliographystyle{agsm}
\bibliography{bibliography.bib}

\end{document}
